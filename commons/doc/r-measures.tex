\label{sec_r_measures}

Сопротивление Образца измеряется 4-х контактным методом, для чего к Образцу подводятся два потенциальных и два токовых контакта. Сопротивление определяется одним из следующих способов:

\subsubsection{Вольтметром/ амперметром}

Используются вольтметр, амперметр и ИП. Амперметр и ИП включены последовательно с Образцом, амперметр подключён к токовым контактам Образца. Вольтметр включён параллельно с Образцом и подключён к его потенциальным контактам. Сопротивление Образца определяется как $R = V/I$, где $R$~--- искомое сопротивление, $V$~--- показания вольтметра, $I$~--- показания амперметра.

Данный способ наиболее универсальный и точный, но требует максимального количества задействованных мультиметров.

\subsubsection{Вольтметром/ вольтметром}

Используются 2 вольтметра, ИП и эталонное сопротивление номинала $R_2$, включённое последовательно с Образцом и ИП. Первый вольтметр подключён  к потенциальным контактам Образца, второй вольметр --- к выходам эталонного сопротивления. Сопротивление Образца определяется как $R = R_2 V_1/V_2$, где $R$~--- искомое сопротивление, $V_1$~--- показания первого вольтметра, $V_2$~--- показания второго вольтметра.

Данный способ требует максимального количества задействованных мультиметров, а также высококачественное эталонное сопротивление. В ряде случаев он может обеспечить повышенную относительную точность измерений, когда сравниваются результаты измерений одного и того же Образца.

\subsubsection{Вольтметром/ вручную}
\label{sec_voltmeter_manually}

Используются вольтметр и ИП, Образец включён последовательно с ИП, вольтметр подключён к потенциальным контактам Образца. Ток в цепи $I$ считается известным  и неизменным в течении всего эксперимента. Сопротивление Образца определяется как $R = V/I$, где $R$~--- искомое сопротивление, $V$~--- показания вольтметра.

Данный способ наименее точный. Если ИП сам не производит измерение тока в цепи, требуется ручное измерение тока.

\subsubsection{Омметром}

Используется омметр, подключённый к Образцу 4-х контактным способом. Сопротивление Образца определяется непосредственным считыванием показаний омметра.

Данный способ наиболее простой, но не годится в том случае, когда сопротивление Образца слишком мало или велико, то есть выходит за диапазон точных измерений омметра.

\bigskip 

Во всех вышеприведённых способах функции вольметра, амперметра или омметра выполняет мультметр, переведённый в соответствующий режим.
