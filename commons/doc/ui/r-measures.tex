\label{sec_r_measure_config}

Как было описано в разделе <<Определение сопротивления>> (стр.~\pageref{sec_r_measures}), Установка может определять сопротивление Образца разными способами. В данном разделе указывается нужный способ и вводятся сопутствующие параметры.

Флаг \CTL{Вольтметром/ Амперметром} включает соответствующий метод измерения. Дополнительных параметров в данном случае нет, и напряжение и ток определяются автоматически.

Флаг \CTL{Вольтметром/ Вольтметром} включает соответствующий метод измерения. В данном режиме необходимо ввести номинал эталонного сопротивления и, желательно, его абсолютную погрешность.

Флаг \CTL{Вольтметром/ вручную} включает соответствующий метод измерения.

Флаг \CTL{Омметром} включает соответствующий метод измерения. Дополнительных параметров в данном случае нет, сопротивление определяется автоматически.

В поле \CTL{Сила тока} нужно ввести силу тока в цепи, запитывающей Образец. Значение данного поля используется следующим образом:

\begin{itemize}
\item Если в качестве I1 используется управляемый источник питания, то при начале измерений ему будет послана команда на установления нужной силы тока. Устанавливать её вручную не требуется.
\item Если используется неуправляемый ИП, то введённое значение используется в режиме \CTL{Вольтметром/вручную} ~--- Программа использует его в качестве известной силы тока.
\end{itemize}

В поле \CTL{Погрешность} вводится известная погрешность ИП в регулировании силы тока. Данное значение используется только в режиме \CTL{Вольтметром/вручную} для определения инструментальной погрешности.

\IMPORTANT{При всяком изменении способа определения сопротивления Программа должна быть перезапущена, чтобы изменения вошли в силу.}