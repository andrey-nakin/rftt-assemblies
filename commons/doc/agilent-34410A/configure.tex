В поле \CTL{Адрес} вводится или выбирается из списка значений адрес устройства. Как правило, если устройство подключено к ПЭВМ и должным образом обнаружено библиотекой VISA, его адрес должен быть в списке.

\bigskip

Если в качестве мультиметра используется Agilent 34401A или аналогичное устройство, подключаемое к ПЭВМ через последовательный интерфейс RS-232, то необходимо указать значения в следующих двух полях.

В поле \CTL{Скорость RS-232} указывается скорость работы последовательного порта. В поле \CTL{Чётность RS-232} указывается способ вычисления контрольной суммы. Эти значения должны совпадать с настройками самого устройства, в противном случае скорее всего связь с ним будет невозможна. Способ настройки см. в документации к мультиметру.

\IMPORTANT{Заводские настройки мультиметра Agilent 34401A следующие: скорость --- \CTL{9600}, чётность --- \CTL{Even}. Эти настройки могут быть изменены, после чего хранятся в энергонезависимой памяти устройства и не теряются при выключении.}

Для мультиметра Agilent 34410A или аналогичного, подключаемого посредством интерфейса USB или GPIB, поля настройки RS-232 должны быть пустыми.

\bigskip

Наконец, поле \CTL{Число циклов 50 Гц на измерение} определяет точность измерений. Мультиметр определяет измеряемую величину не одномоментно, а путём многократных измерений в течении некоторого времени. Если это время кратно продолжительности периода питания сети 220~В, то на результат измерений не будет влиять шум вызванный колебанием питания. Время каждого измерения определяется по формуле:

\begin{equation}
t = N_{50} \cdot 20 \cdot 2,
\end{equation}

\noindent где $t$~--- время в мс, $N_{50}$~--- число циклов, $20$~--- продолжительность одного периода при частоте 50~Гц, $2$~--- коэффициент вызванный тем, что каждое измерение фактически производится дважды: первый раз оно производится для измерения <<нуля>>. Такой способ измерений предотвращает <<уход нуля>> при продолжительных измерениях. Таким образом, при $N_{50} = 1$ продолжительность одного измерения равна $40$~мс, а при $N_{50} = 100$ --- $4$~с.

Согласно документации к мультиметру, при увеличении $N_{50}$ (там этот параметр называется NPLC~--- Number of Power Line Cycles) увеличивается точность измерения. Так, например, при увеличении $N_{50}$ от 1 до 100 абсолютная погрешность измерения постоянного напряжения уменьшается на $\approx 1.5\%$. С другой стороны увеличение $N_{50}$ также увеличивает время измерения, что может быть критичным при высоких скоростях изменения температуры. Подробную информацию о влиянии $N_{50}$ на точность измерений см. в документации к мультиметру.

Рекомендуемое значение параметра~--- 10. Не рекомендуется выбирать в качестве числа циклов дробное значение, поскольку в данном случае результаты измерений будут содержать заметный шум на частоте $50$~Гц.

\bigskip

Для проверки правильности параметров можно установить пробное соединение с устройством, для этого нажмите кнопку \CTL{Опрос}. Программа попытается установить связь, после чего сообщит о результатах. Не рекомендуется производить опрос в процессе измерений, это может привести к потере данных.