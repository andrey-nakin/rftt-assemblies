Программа записывает информацию о всех обнаруженных неисправностях аппаратной части в файл протокола с именем \FILENAME{\PROGNAME{}.log}, который размещается в текущем каталоге. Файл всегда пополняется, то есть при обнаружении ошибки новые записи добавляются в конец файла.

Файл имеет простой текстовый формат, в котором каждая запись имеет следующий вид:

\CMD{Время Важность Модуль Описание}

Здесь <Время>~--- точные дата и время обнаружения неисправности. <Важность>~--- важность ситуации, которая может принимать следующие значения:

\begin{itemize}
\item \CTL{critical} --- критическая ошибка;
\item \CTL{error} --- важная ошибка;
\item \CTL{warning} -- предупреждение о возможной ошибке;
\item \CTL{info} --- информационное сообщение, не сигнализирующее об ошибке;
\item \CTL{debug} --- отладочное сообщение, предназначенное для отладки программы.
\end{itemize}

<Модуль>~--- имя модуля Программы, в котором обнаружена ошибка. <Описание>~--- произвольное текстовое описание, детально раскрывающее суть и местоположение ошибки. Описание может распологаться на нескольких строках файла протокола.

Если файл протокола отсутствует при работающей программе, значит в процессе работы ещё не возникало ни одной ошибки.

Файл протокола можно безопасно удалять, он будет вновь создан при первой же возникшей ошибке.

Поскольку файл протокола постоянно пополняется, оператору следует время от времени удалять его во избежание переполнения диска.
