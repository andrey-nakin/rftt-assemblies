В поле \CTL{Адрес} вводится или выбирается из списка значений адрес устройства. Как правило, если устройство подключено к ПЭВМ и должным образом обнаружено библиотекой VISA, его адрес должен быть в списке.

В поле \CTL{Скорость RS-232} указывается скорость работы последовательного порта. В поле \CTL{Чётность RS-232} указывается способ вычисления контрольной суммы. Эти значения должны совпадать с настройками самого устройства, в противном случае скорее всего связь с ним будет невозможна. Способ настройки см. в документации к устройству.

\IMPORTANT{Заводские настройки Agilent E3645A следующие: скорость --- \CTL{9600}, чётность --- \CTL{None}. Эти настройки могут быть изменены, после чего хранятся в энергонезависимой памяти устройства и не теряются при выключении.}

В поле \CTL{Максимальный ток} указывается максимально возможный ток, который может быть подан в цепь. Поле может быть пустым.

\bigskip

Для проверки правильности параметров можно установить пробное соединение с устройством, для этого нажмите кнопку \CTL{Опрос}. Программа попытается установить связь, после чего сообщит о результатах. Не рекомендуется производить опрос в процессе измерений, это может привести к потере данных.