Если возможно, введите некоторые геометрические параметры. Они будут использоваться для вычисления удельного сопротивления Образца. 

В поле \CTL{Расстояние между потенциальными контактами} введите соответствующее расстояние и его погрешность. В самом простом случае этого достаточно для вычисления удельного сопротивления.

Если Образец имеет форму близкую к параллелепипеду, то введите соответствующие размеры в полях \CTL{Длина}, \CTL{Ширина} и \CTL{Толщина}, а также их абсолютные погрешности. При этом длина Образца не используется в расчётах, а ширина и толщина используются для расчёта поперечного сечения.

Если Образец имеет форму, близкую к цилиндрической, то заполните поле \CTL{Диаметр} и, по возможности, поле для абсолютной погрешности. Диаметр будет использоваться для расчёта поперечного сечения и, в конечном итоге, удельного сопротивления. Если указан диаметр, то ширину и толщину Образца вводить не нужно.

Допускается, хотя и не рекомендуется, не указывать погрешности геометрических параметров. В этом случае соответствующие поля ввода должны быть пустыми.

Если геометрические параметры Образца неизвестны, или Образец имеет неправильную форму, то очистите все соответствующие поля. В этом случае удельное сопротивление не будет вычисляться автоматически, но на общую работу установки это не повлияет.

См. раздел <<Удельное сопротивление>> на стр.~\pageref{sec_rho_calcs} с формулами расчёта удельного сопротивления.