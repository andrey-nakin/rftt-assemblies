\documentclass[12pt, a4paper]{article}
\usepackage[utf8]{inputenc}
\usepackage[russian]{babel}
\usepackage{hyperref}
\usepackage[]{graphicx}

\title{Коммутационный блок №~2. \\ Инструкция по электромонтажу}
\author{Накин~А.~В.}

\newcommand{\PIN}[1]{{\bf #1}}

\begin{document}

\maketitle

\section{Общие положения}

Необходимо произвести электромонтаж компонентов коммутационного блока №~2 (далее --- Блока). Монтаж Блока осуществляется согласно приложенной принципиальной схеме.

Состав Блока:

\begin{itemize}
\item Корпус пластиковый.
\item Разъём $J1$ с 8 прижимными контактами в корпусе.
\item Разъём $J4$ типа DSUB-9 с 9 контактами.
\item Разъём $J5$ типа DSUB-9 с 9 контактами.
\item Разъёмы $J4$ и $J5$ конструктивно объединены в одном корпусе.
\item Разъём $J2$ с 3 контактами для питания 220~В.
\item Прибор МВУ-8. Прибор конструктивно объединяет в себе переключатели $S1$--$S5$, обозначенные на принципиальной схеме.
\item Прибор АС-4.
\item Прибор ТРМ-201.
\item Двухполюсной автоматический выключатель $S6$.
\item Электрические провода.
\end{itemize}

Следующие компоненты, обозначенные на принципиальной схеме, не входят в состав Блока, а подключаются к внешним разъёмам и не требуют монтажа:

\begin{itemize}
\item Мультиметры $MM1$, $MM2$.
\item Измеряемый образец $R1$.
\item Эталонное сопротивление $R2$.
\item Источник питания $I1$.
\item Термопара $TC$.
\end{itemize}


\section{Способ монтажа}
\label{sec_wires}

\begin{itemize}
\item Выходы \PIN{A}, \PIN{B} МВУ-8 соединяются с выходами  \PIN{A}, \PIN{B} АС-4 витой парой, желательно экранированной.
\item Выходы \PIN{A}, \PIN{B} ТРМ-201 соединяются с выходами  \PIN{A}, \PIN{B} АС-4 витой парой, желательно экранированной.
\item Выходы \PIN{10}, \PIN{11} ТРМ-201 соединяются с выходами \PIN{3}, \PIN{4} разъёма $J1$ многожильным изолированным медным проводом произвольного сечения.
\item Цепи питания 220~В выполнена многожильным изолированным медным проводом сечения 0,75~мм${}^2$ или выше.
\item Прочие электрические соединения выполнена многожильным изолированным медным проводом такого сечения, чтобы оно допускало постоянный ток не менее 3~А.
\end{itemize}

\section{Порядок монтажа}

\begin{enumerate}

\item Корпус Блока состоит имеет крышку, крепящуюся на 4 саморезах. Электромонтаж Блока должен быть проведён таким образом, чтобы крышку можно было отвернуть на 180 градусов без нарушения монтажа.

\item Выполните распайку разъёма $J1$ в корпусе таким образом, чтобы каждый прижимной контакт на передней панели был подключен к соответствующему винтовому контакту на контактной плате. Провода для электромонтажа выбираются согласно разделу \ref{sec_wires}.

\item Выполните распайку разъёмов $J4$ и $J5$ в корпусе таким образом, чтобы каждый контакт разъёма панели был подключен к соответствующему винтовому контакту на контактной плате. Провода для электромонтажа выбираются согласно разделу \ref{sec_wires}.

\item Установите компоненты Блока на DIN рейки корпуса:

\begin{itemize}
\item Верхняя DIN рейка: МВУ-8, АС-4.
\item Средняя DIN рейка: ТРМ-201 (не прикрепляется к рейке).
\item Нижняя DIN рейка: Корпус разъёма $J1$, корпус разъёмов $J4$/$J5$, выключатель $S6$.
\end{itemize}

\item Зафиксируйте компоненты на DIN рейках прижимными ограничителями. ТРМ-201 не фиксируется на DIN рейке.

\item Выполните соединение компонентов согласно принципиальной схемы. Провода для электромонтажа выбираются согласно разделу \ref{sec_wires}.

\end{enumerate}

\end{document}
